% Options for packages loaded elsewhere
\PassOptionsToPackage{unicode}{hyperref}
\PassOptionsToPackage{hyphens}{url}
\documentclass[
]{article}
\usepackage{xcolor}
\usepackage[margin=1in]{geometry}
\usepackage{amsmath,amssymb}
\setcounter{secnumdepth}{-\maxdimen} % remove section numbering
\usepackage{iftex}
\ifPDFTeX
  \usepackage[T1]{fontenc}
  \usepackage[utf8]{inputenc}
  \usepackage{textcomp} % provide euro and other symbols
\else % if luatex or xetex
  \usepackage{unicode-math} % this also loads fontspec
  \defaultfontfeatures{Scale=MatchLowercase}
  \defaultfontfeatures[\rmfamily]{Ligatures=TeX,Scale=1}
\fi
\usepackage{lmodern}
\ifPDFTeX\else
  % xetex/luatex font selection
\fi
% Use upquote if available, for straight quotes in verbatim environments
\IfFileExists{upquote.sty}{\usepackage{upquote}}{}
\IfFileExists{microtype.sty}{% use microtype if available
  \usepackage[]{microtype}
  \UseMicrotypeSet[protrusion]{basicmath} % disable protrusion for tt fonts
}{}
\makeatletter
\@ifundefined{KOMAClassName}{% if non-KOMA class
  \IfFileExists{parskip.sty}{%
    \usepackage{parskip}
  }{% else
    \setlength{\parindent}{0pt}
    \setlength{\parskip}{6pt plus 2pt minus 1pt}}
}{% if KOMA class
  \KOMAoptions{parskip=half}}
\makeatother
\usepackage{color}
\usepackage{fancyvrb}
\newcommand{\VerbBar}{|}
\newcommand{\VERB}{\Verb[commandchars=\\\{\}]}
\DefineVerbatimEnvironment{Highlighting}{Verbatim}{commandchars=\\\{\}}
% Add ',fontsize=\small' for more characters per line
\usepackage{framed}
\definecolor{shadecolor}{RGB}{248,248,248}
\newenvironment{Shaded}{\begin{snugshade}}{\end{snugshade}}
\newcommand{\AlertTok}[1]{\textcolor[rgb]{0.94,0.16,0.16}{#1}}
\newcommand{\AnnotationTok}[1]{\textcolor[rgb]{0.56,0.35,0.01}{\textbf{\textit{#1}}}}
\newcommand{\AttributeTok}[1]{\textcolor[rgb]{0.13,0.29,0.53}{#1}}
\newcommand{\BaseNTok}[1]{\textcolor[rgb]{0.00,0.00,0.81}{#1}}
\newcommand{\BuiltInTok}[1]{#1}
\newcommand{\CharTok}[1]{\textcolor[rgb]{0.31,0.60,0.02}{#1}}
\newcommand{\CommentTok}[1]{\textcolor[rgb]{0.56,0.35,0.01}{\textit{#1}}}
\newcommand{\CommentVarTok}[1]{\textcolor[rgb]{0.56,0.35,0.01}{\textbf{\textit{#1}}}}
\newcommand{\ConstantTok}[1]{\textcolor[rgb]{0.56,0.35,0.01}{#1}}
\newcommand{\ControlFlowTok}[1]{\textcolor[rgb]{0.13,0.29,0.53}{\textbf{#1}}}
\newcommand{\DataTypeTok}[1]{\textcolor[rgb]{0.13,0.29,0.53}{#1}}
\newcommand{\DecValTok}[1]{\textcolor[rgb]{0.00,0.00,0.81}{#1}}
\newcommand{\DocumentationTok}[1]{\textcolor[rgb]{0.56,0.35,0.01}{\textbf{\textit{#1}}}}
\newcommand{\ErrorTok}[1]{\textcolor[rgb]{0.64,0.00,0.00}{\textbf{#1}}}
\newcommand{\ExtensionTok}[1]{#1}
\newcommand{\FloatTok}[1]{\textcolor[rgb]{0.00,0.00,0.81}{#1}}
\newcommand{\FunctionTok}[1]{\textcolor[rgb]{0.13,0.29,0.53}{\textbf{#1}}}
\newcommand{\ImportTok}[1]{#1}
\newcommand{\InformationTok}[1]{\textcolor[rgb]{0.56,0.35,0.01}{\textbf{\textit{#1}}}}
\newcommand{\KeywordTok}[1]{\textcolor[rgb]{0.13,0.29,0.53}{\textbf{#1}}}
\newcommand{\NormalTok}[1]{#1}
\newcommand{\OperatorTok}[1]{\textcolor[rgb]{0.81,0.36,0.00}{\textbf{#1}}}
\newcommand{\OtherTok}[1]{\textcolor[rgb]{0.56,0.35,0.01}{#1}}
\newcommand{\PreprocessorTok}[1]{\textcolor[rgb]{0.56,0.35,0.01}{\textit{#1}}}
\newcommand{\RegionMarkerTok}[1]{#1}
\newcommand{\SpecialCharTok}[1]{\textcolor[rgb]{0.81,0.36,0.00}{\textbf{#1}}}
\newcommand{\SpecialStringTok}[1]{\textcolor[rgb]{0.31,0.60,0.02}{#1}}
\newcommand{\StringTok}[1]{\textcolor[rgb]{0.31,0.60,0.02}{#1}}
\newcommand{\VariableTok}[1]{\textcolor[rgb]{0.00,0.00,0.00}{#1}}
\newcommand{\VerbatimStringTok}[1]{\textcolor[rgb]{0.31,0.60,0.02}{#1}}
\newcommand{\WarningTok}[1]{\textcolor[rgb]{0.56,0.35,0.01}{\textbf{\textit{#1}}}}
\usepackage{graphicx}
\makeatletter
\newsavebox\pandoc@box
\newcommand*\pandocbounded[1]{% scales image to fit in text height/width
  \sbox\pandoc@box{#1}%
  \Gscale@div\@tempa{\textheight}{\dimexpr\ht\pandoc@box+\dp\pandoc@box\relax}%
  \Gscale@div\@tempb{\linewidth}{\wd\pandoc@box}%
  \ifdim\@tempb\p@<\@tempa\p@\let\@tempa\@tempb\fi% select the smaller of both
  \ifdim\@tempa\p@<\p@\scalebox{\@tempa}{\usebox\pandoc@box}%
  \else\usebox{\pandoc@box}%
  \fi%
}
% Set default figure placement to htbp
\def\fps@figure{htbp}
\makeatother
\setlength{\emergencystretch}{3em} % prevent overfull lines
\providecommand{\tightlist}{%
  \setlength{\itemsep}{0pt}\setlength{\parskip}{0pt}}
\usepackage{bookmark}
\IfFileExists{xurl.sty}{\usepackage{xurl}}{} % add URL line breaks if available
\urlstyle{same}
\hypersetup{
  pdfauthor={Sara Ji and Bailey Buchanan},
  hidelinks,
  pdfcreator={LaTeX via pandoc}}

\title{for Applied Educational Research

Spring 2026

Week 3 Data Analytic Memo Starter Code

S-052: Intermediate and Advanced Statistical Methods}
\author{Sara Ji and Bailey Buchanan}
\date{Last Modified: Feb 2026}

\begin{document}
\maketitle

\subsubsection{RQ: What is the causal effect of technology-led
afterschool on math
scores?}\label{rq-what-is-the-causal-effect-of-technology-led-afterschool-on-math-scores}

\subsection{Load in packages}\label{load-in-packages}

\begin{Shaded}
\begin{Highlighting}[]
\CommentTok{\# These packages will be helpful for your analysis}
\FunctionTok{library}\NormalTok{(foreign)    }\CommentTok{\# For reading various data formats}
\FunctionTok{library}\NormalTok{(haven)      }\CommentTok{\# For reading Stata files}
\FunctionTok{library}\NormalTok{(tidyverse)  }\CommentTok{\# For data manipulation and visualization}
\FunctionTok{library}\NormalTok{(stargazer)  }\CommentTok{\# For nice regression tables}
\FunctionTok{require}\NormalTok{(gridExtra)}
\CommentTok{\# If you haven\textquotesingle{}t installed these packages yet, run:}
\CommentTok{\# install.packages(c("foreign", "haven", "tidyverse", "stargazer"))}
\end{Highlighting}
\end{Shaded}

\subsection{Define the working directory and load
data}\label{define-the-working-directory-and-load-data}

\begin{Shaded}
\begin{Highlighting}[]
\CommentTok{\# Set your working directory to where your data file is located}
\CommentTok{\# Update the path below to match your file location}
\FunctionTok{setwd}\NormalTok{(}\StringTok{"/Users/juhong/Desktop/S052/Data"}\NormalTok{)}
\end{Highlighting}
\end{Shaded}

\#\#Data: Thanks to Prof.~Alejandro Ganimian and coauthors Karthik
Muralidharan and Abhijeet Singh for this data

\begin{Shaded}
\begin{Highlighting}[]
\CommentTok{\# Load the data}
\NormalTok{dat }\OtherTok{\textless{}{-}} \FunctionTok{read\_dta}\NormalTok{(}\StringTok{"/Users/juhong/Desktop/S052/W3/mindspark\_data.dta"}\NormalTok{)}
\NormalTok{origin }\OtherTok{\textless{}{-}} \FunctionTok{read\_dta}\NormalTok{(}\StringTok{"/Users/juhong/Desktop/S052/W3/mindspark\_data.dta"}\NormalTok{)}
\end{Highlighting}
\end{Shaded}

\subsection{Q1: Exploratory Data
Analysis}\label{q1-exploratory-data-analysis}

\begin{Shaded}
\begin{Highlighting}[]
\CommentTok{\# For this question, you will need:}
\CommentTok{\# {-} The hist() function or ggplot() with geom\_histogram() for histograms}
\CommentTok{\# {-} The summarize() function from dplyr or the summary() function for descriptive statistics}
\CommentTok{\# {-} You can calculate statistics by group using group\_by() and summarize() in dplyr}
\CommentTok{\# {-} Or use aggregate() or tapply() in base R}
\CommentTok{\# }
\CommentTok{\# Hint: To get means by groups, you might use something like:}
\CommentTok{\# dat \%\textgreater{}\%}
   \CommentTok{\# group\_by(variable1, variable2) \%\textgreater{}\%}
   \CommentTok{\# summarize(mean\_score = mean(math\_std, na.rm = TRUE))}

\CommentTok{\# Before turning to Google, AI, etc., try typing \textasciigrave{}?function\_name\textasciigrave{} in the console to see documentation}
\CommentTok{\# {-} Example: \textasciigrave{}?lm\textasciigrave{} for help with linear models}
\end{Highlighting}
\end{Shaded}

sum(is.na(dat)) dat \textless- drop\_na(dat)

dat\$female \%\textgreater\% table()

head(dat) 413/(124+413)\emph{100 470/(149+470)}100

counts\_table \textless- table(dat\$female)
print(prop.table(counts\_table, margin = 1))

dat \%\textgreater\% group\_by(female, treat) \%\textgreater\%
summarize(mean\_score = mean(math\_std,na.rm = TRUE))

\subsubsection{v1: Histogram of math scores on gender and
treatment}\label{v1-histogram-of-math-scores-on-gender-and-treatment}

dat\(female_f <- factor(dat\)female, levels = c(0, 1), labels =
c(``Male'', ``Female''))

dat\(treat_f <- factor(dat\)treat, levels = c(0, 1), labels =
c(``Control'', ``Treatment''))

gender\_hist \textless- ggplot(dat, aes(x = math\_std)) +
geom\_histogram(fill = ``light blue'', \#Set bar fill color, color =
``black'') + \#Set bar border color
facet\_wrap(\textasciitilde female\_f)

treat\_hist \textless- ggplot(dat, aes(x = math\_std)) +
geom\_histogram(fill = ``light pink'', color = ``black'') +
facet\_wrap(\textasciitilde treat\_f)

grid.arrange(grobs = list(gender\_hist,treat\_hist))

ggplot(dat, aes(x = female, y= math\_std)) + geom\_point(fill = ``light
blue'', color = ``black'')

\subsubsection{v2: Histogram per group}\label{v2-histogram-per-group}

f\_t \textless- dat\%\textgreater\%filter(female == 1, treat == 1)
\%\textgreater\%
ggplot(aes(x=math\_std))+geom\_histogram()+ggtitle(``Female in
Treated'') f\_c \textless- dat\%\textgreater\%filter(female == 1, treat
== 0) \%\textgreater\%
ggplot(aes(x=math\_std))+geom\_histogram()+ggtitle(``Female in
Controlled'') m\_t \textless- dat\%\textgreater\%filter(female == 0,
treat == 1) \%\textgreater\%
ggplot(aes(x=math\_std))+geom\_histogram()+ggtitle(``Male in Treated'')
m\_c \textless- dat\%\textgreater\%filter(female == 0, treat == 0)
\%\textgreater\%
ggplot(aes(x=math\_std))+geom\_histogram()+ggtitle(``Male in
Controlled'')

grid.arrange(grobs = list(f\_t, f\_c, m\_t, m\_c))

\subsubsection{V3: Histogram of grouped by gender, colored by treat,
counts
percentaged}\label{v3-histogram-of-grouped-by-gender-colored-by-treat-counts-percentaged}

f\_t\_dat \textless- dat\%\textgreater\%filter(female == 1, treat == 1)
f\_t \textless- f\_t\_dat \%\textgreater\%ggplot(aes(x=math\_std, y=
after\_stat(100*count/ sum(count))))+geom\_histogram()+ggtitle(``Female
in Treated'')+ xlim(-5, 5) f\_c \textless-
dat\%\textgreater\%filter(female == 1, treat == 0) \%\textgreater\%
ggplot(aes(x=math\_std))+geom\_histogram()+ggtitle(``Female in
Controlled'')+ xlim(-5, 5) m\_t \textless-
dat\%\textgreater\%filter(female == 0, treat == 1) \%\textgreater\%
ggplot(aes(x=math\_std))+geom\_histogram()+ggtitle(``Male in Treated'')+
xlim(-5, 5) m\_c \textless- dat\%\textgreater\%filter(female == 0, treat
== 0) \%\textgreater\%
ggplot(aes(x=math\_std))+geom\_histogram()+ggtitle(``Male in
Controlled'')+ xlim(-5, 5)

grid.arrange(grobs = list(f\_t, f\_c, m\_t, m\_c))

\subsubsection{Final version: Histogram grouped by gender, colored by
treat}\label{final-version-histogram-grouped-by-gender-colored-by-treat}

library(dplyr) library(ggplot2) library(gridExtra)

m\_all \textless- dat \%\textgreater\% filter(female == 0)
\%\textgreater\% ggplot(aes(x = math\_std, fill = treat\_f, group =
treat\_f)) + \# group by treat geom\_histogram( aes(y = after\_stat(100
* count / sum(count))), \# \% within each (male, treat) position =
``identity'', alpha = 0.6 ) + ggtitle(``Male'') + labs(fill = ``Group'',
y = ``Percent for each group'')+xlim(-4, 4)+ylim(0, 8)

f\_all \textless- dat \%\textgreater\% filter(female == 1)
\%\textgreater\% ggplot(aes(x = math\_std, fill = treat\_f, group =
treat\_f)) + geom\_histogram( aes(y = after\_stat(100 * count /
sum(count))), \# \% within each (female, treat) position = ``identity'',
alpha = 0.6 ) + ggtitle(``Female'') + labs(fill = ``Group'', y =
``Percent for each group'')+xlim(-4, 4)+ylim(0, 8)

grid.arrange(grobs = list(f\_all, m\_all))

library(e1071)

origin \textless- drop\_na(origin)

f\_t\_dat \textless- origin\%\textgreater\%filter(female == 1, treat ==
1) f\_c\_dat \textless- origin\%\textgreater\%filter(female == 1, treat
== 0) m\_t\_dat \textless- origin\%\textgreater\%filter(female == 0,
treat == 1) m\_c\_dat \textless- origin\%\textgreater\%filter(female ==
0, treat == 0)

dat \%\textgreater\% group\_by(female, treat) \%\textgreater\%
summarize(skewness = skewness(math\_std,na.rm = TRUE)) dat
\%\textgreater\% group\_by(female, treat) \%\textgreater\%
summarize(mean\_score = mean(math\_std,na.rm = TRUE)) dat
\%\textgreater\% group\_by(female, treat) \%\textgreater\%
summarize(standard\_deviation = sd(math\_std,na.rm = TRUE))

\subsection{Q2: Separate and joint estimation of causal
effects}\label{q2-separate-and-joint-estimation-of-causal-effects}

\begin{Shaded}
\begin{Highlighting}[]
\CommentTok{\# Convert variables to factors if needed (this helps R handle categorical variables properly)}
\NormalTok{dat}\SpecialCharTok{$}\NormalTok{treat }\OtherTok{\textless{}{-}} \FunctionTok{as\_factor}\NormalTok{(dat}\SpecialCharTok{$}\NormalTok{treat)}
\NormalTok{dat}\SpecialCharTok{$}\NormalTok{female }\OtherTok{\textless{}{-}} \FunctionTok{as\_factor}\NormalTok{(dat}\SpecialCharTok{$}\NormalTok{female)}

\CommentTok{\# For this question, you will need the lm() function for linear regression}
\CommentTok{\# Remember to use robust standard errors!}
\NormalTok{m\_all\_dat }\OtherTok{\textless{}{-}}\NormalTok{ dat }\SpecialCharTok{\%\textgreater{}\%}\FunctionTok{filter}\NormalTok{(female }\SpecialCharTok{==} \DecValTok{0}\NormalTok{)}
\CommentTok{\# Model 1: Effect for male students only}
\CommentTok{\# Hint: Use filter() or subset to select only male students (female == "0")}
\CommentTok{\# lm(math\_std \textasciitilde{} treat, data = ...)}
\FunctionTok{head}\NormalTok{(m\_all\_dat)}
\end{Highlighting}
\end{Shaded}

\begin{verbatim}
## # A tibble: 6 x 4
##   st_id treat female math_std
##   <chr> <fct> <fct>     <dbl>
## 1 GP001 0     0        1.72  
## 2 GP002 1     0        0.599 
## 3 GP007 1     0       -0.0312
## 4 GP008 0     0        0.856 
## 5 GP009 0     0        0.802 
## 6 GP010 1     0        0.916
\end{verbatim}

\begin{Shaded}
\begin{Highlighting}[]
\NormalTok{mod1 }\OtherTok{\textless{}{-}} \FunctionTok{lm}\NormalTok{(math\_std }\SpecialCharTok{\textasciitilde{}}\NormalTok{ treat, }\AttributeTok{data =}\NormalTok{ m\_all\_dat)}\CommentTok{\# Your code here}
   
\CommentTok{\# If you want to see the full model output, including coefficient estimates, standard errors, test statistics, and p{-}values, call summary(mod).}
\FunctionTok{summary}\NormalTok{(mod1)}
\end{Highlighting}
\end{Shaded}

\begin{verbatim}
## 
## Call:
## lm(formula = math_std ~ treat, data = m_all_dat)
## 
## Residuals:
##      Min       1Q   Median       3Q      Max 
## -3.14930 -0.47993  0.06435  0.61717  1.75570 
## 
## Coefficients:
##             Estimate Std. Error t value Pr(>|t|)   
## (Intercept)   0.3346     0.1054   3.175   0.0019 **
## treat1        0.4228     0.1515   2.791   0.0061 **
## ---
## Signif. codes:  0 '***' 0.001 '**' 0.01 '*' 0.05 '.' 0.1 ' ' 1
## 
## Residual standard error: 0.8431 on 122 degrees of freedom
##   (25 observations deleted due to missingness)
## Multiple R-squared:  0.06001,    Adjusted R-squared:  0.0523 
## F-statistic: 7.788 on 1 and 122 DF,  p-value: 0.006105
\end{verbatim}

\begin{Shaded}
\begin{Highlighting}[]
\CommentTok{\# Model 2: Effect for female students only}
\NormalTok{f\_all\_dat }\OtherTok{\textless{}{-}}\NormalTok{ dat}\SpecialCharTok{\%\textgreater{}\%}\FunctionTok{filter}\NormalTok{(female }\SpecialCharTok{==} \DecValTok{1}\NormalTok{)}
\NormalTok{mod2 }\OtherTok{\textless{}{-}} \FunctionTok{lm}\NormalTok{(math\_std }\SpecialCharTok{\textasciitilde{}}\NormalTok{ treat, }\AttributeTok{data =}\NormalTok{ f\_all\_dat)}
\FunctionTok{summary}\NormalTok{(mod2)}
\end{Highlighting}
\end{Shaded}

\begin{verbatim}
## 
## Call:
## lm(formula = math_std ~ treat, data = f_all_dat)
## 
## Residuals:
##     Min      1Q  Median      3Q     Max 
## -4.2424 -0.5564  0.0108  0.6537  2.5031 
## 
## Coefficients:
##             Estimate Std. Error t value Pr(>|t|)    
## (Intercept)  0.31839    0.06757   4.712 3.37e-06 ***
## treat1       0.32757    0.09591   3.415 0.000701 ***
## ---
## Signif. codes:  0 '***' 0.001 '**' 0.01 '*' 0.05 '.' 0.1 ' ' 1
## 
## Residual standard error: 0.9746 on 411 degrees of freedom
##   (57 observations deleted due to missingness)
## Multiple R-squared:  0.0276, Adjusted R-squared:  0.02523 
## F-statistic: 11.66 on 1 and 411 DF,  p-value: 0.0007005
\end{verbatim}

\begin{Shaded}
\begin{Highlighting}[]
\CommentTok{\# Model 3: Full model with interaction}
\CommentTok{\# Hint: Use the formula: math\_std \textasciitilde{} treat*female}
\CommentTok{\# The * automatically includes main effects and interaction}
\NormalTok{mod3 }\OtherTok{\textless{}{-}} \FunctionTok{lm}\NormalTok{(math\_std }\SpecialCharTok{\textasciitilde{}}\NormalTok{ treat}\SpecialCharTok{*}\NormalTok{female, }\AttributeTok{data=}\NormalTok{dat)}
\FunctionTok{summary}\NormalTok{(mod3)}
\end{Highlighting}
\end{Shaded}

\begin{verbatim}
## 
## Call:
## lm(formula = math_std ~ treat * female, data = dat)
## 
## Residuals:
##     Min      1Q  Median      3Q     Max 
## -4.2424 -0.5549  0.0240  0.6503  2.5031 
## 
## Coefficients:
##                Estimate Std. Error t value Pr(>|t|)   
## (Intercept)     0.33460    0.11826   2.829  0.00484 **
## treat1          0.42283    0.17001   2.487  0.01319 * 
## female1        -0.01621    0.13524  -0.120  0.90465   
## treat1:female1 -0.09526    0.19384  -0.491  0.62333   
## ---
## Signif. codes:  0 '***' 0.001 '**' 0.01 '*' 0.05 '.' 0.1 ' ' 1
## 
## Residual standard error: 0.9461 on 533 degrees of freedom
##   (82 observations deleted due to missingness)
## Multiple R-squared:  0.03429,    Adjusted R-squared:  0.02885 
## F-statistic: 6.308 on 3 and 533 DF,  p-value: 0.0003288
\end{verbatim}

\begin{Shaded}
\begin{Highlighting}[]
\CommentTok{\# This code will output your regression results into a nice table}
\CommentTok{\# (Similar to esttab in Stata)}
\FunctionTok{stargazer}\NormalTok{(mod1, mod2, mod3, }
          \AttributeTok{type =} \StringTok{"text"}\NormalTok{,}
          \AttributeTok{column.labels =} \FunctionTok{c}\NormalTok{(}\StringTok{"Male"}\NormalTok{, }\StringTok{"Female"}\NormalTok{, }\StringTok{"Full"}\NormalTok{),}
          \AttributeTok{title =} \StringTok{"...WRITE A TITLE HERE..."}\NormalTok{,}
          \AttributeTok{star.char =} \FunctionTok{c}\NormalTok{(}\StringTok{"*"}\NormalTok{, }\StringTok{"**"}\NormalTok{, }\StringTok{"***"}\NormalTok{),}
          \AttributeTok{star.cutoffs =} \FunctionTok{c}\NormalTok{(}\FloatTok{0.05}\NormalTok{, }\FloatTok{0.01}\NormalTok{, }\FloatTok{0.001}\NormalTok{),}
          \AttributeTok{notes =} \FunctionTok{c}\NormalTok{(}\StringTok{"*p\textless{}0.05; **p\textless{}0.01; ***p\textless{}0.001"}\NormalTok{),}
          \AttributeTok{notes.append =} \ConstantTok{FALSE}\NormalTok{)}
\end{Highlighting}
\end{Shaded}

\begin{verbatim}
## 
## ...WRITE A TITLE HERE...
## ========================================================================================
##                                             Dependent variable:                         
##                     --------------------------------------------------------------------
##                                                   math_std                              
##                             Male                  Female                   Full         
##                              (1)                    (2)                    (3)          
## ----------------------------------------------------------------------------------------
## treat1                     0.423**               0.328***                 0.423*        
##                            (0.152)                (0.096)                (0.170)        
##                                                                                         
## female1                                                                   -0.016        
##                                                                          (0.135)        
##                                                                                         
## treat1:female1                                                            -0.095        
##                                                                          (0.194)        
##                                                                                         
## Constant                   0.335**               0.318***                0.335**        
##                            (0.105)                (0.068)                (0.118)        
##                                                                                         
## ----------------------------------------------------------------------------------------
## Observations                 124                    413                    537          
## R2                          0.060                  0.028                  0.034         
## Adjusted R2                 0.052                  0.025                  0.029         
## Residual Std. Error   0.843 (df = 122)       0.975 (df = 411)        0.946 (df = 533)   
## F Statistic         7.788** (df = 1; 122) 11.664*** (df = 1; 411) 6.308*** (df = 3; 533)
## ========================================================================================
## Note:                                                      *p<0.05; **p<0.01; ***p<0.001
\end{verbatim}

\subsection{Coding Pathway}\label{coding-pathway}

library(sjPlot) library(sjmisc) library(ggplot2)
theme\_set(theme\_sjplot()) plot\_model(mod3, type=``int'')

library(dplyr)

dat2 \textless- dat \%\textgreater\% mutate(group = case\_when( treat ==
1 \& female == 1 \textasciitilde{} ``Female Treated'', treat == 1 \&
female == 0 \textasciitilde{} ``Male Treated'', treat == 0 \& female ==
1 \textasciitilde{} ``Female Control'', treat == 0 \& female == 0
\textasciitilde{} ``Male Control'' ))

ggplot(dat2, aes(x = factor(group), y = math\_std, color =
factor(treat))) + geom\_jitter(alpha = 1, position =
position\_dodge(width = 0.5)) + stat\_summary(fun = mean, geom =
``point'', size = 4, position = position\_dodge(width = 0.5)) +
stat\_summary( fun = mean, fun.min = function(x) mean(x) -
sd(x)/sqrt(length(x)), fun.max = function(x) mean(x) +
sd(x)/sqrt(length(x)), geom = ``errorbar'', width = 0.2, position =
position\_dodge(width = 0.5) ) + theme\_minimal() + labs( title =
``Endline Math Scores by Group and Treatment'', y = ``Standardized Math
Score'', x = ``Group'' ) + ylim(-4, 4) + scale\_color\_manual( values =
c(``0'' = ``lightblue'', ``1'' = ``pink''), guide = ``none'' )

\subsection{Additional Notes for
Students:}\label{additional-notes-for-students}

\subsubsection{Helpful R Tips:}\label{helpful-r-tips}

\begin{enumerate}
\def\labelenumi{\arabic{enumi}.}
\item
  \textbf{Filtering data}: Use \texttt{filter()} from dplyr

  \begin{itemize}
  \tightlist
  \item
    Example: \texttt{filter(dat,\ female\ ==\ 0)}
  \end{itemize}
\item
  \textbf{Creating groups}: For analysis by groups, you can:

  \begin{itemize}
  \tightlist
  \item
    Use \texttt{group\_by()} and \texttt{summarize()} from dplyr
  \item
    Use the \texttt{by()} function in base R
  \item
    Filter and analyze separately
  \end{itemize}
\item
  \textbf{Getting help}: Type \texttt{?function\_name} in the console to
  see documentation

  \begin{itemize}
  \tightlist
  \item
    Example: \texttt{?lm} for help with linear models
  \end{itemize}
\item
  \textbf{Checking your work}: Use \texttt{summary()} on your model
  objects to see detailed output
\item
  \textbf{For the interaction model}: Remember that
  \texttt{treat*female} in the formula automatically includes both main
  effects and the interaction term
\end{enumerate}

\end{document}
